\title{CS289 Initial Report: \\
Methods of handling missing data for classification
\vspace{15mm}}

\author{Rafael Valle\thanks{\href{mailto:rafaelvalle@berkeley.com}{\nolinkurl{rafaelvalle@berkeley.com}}.} \hspace{10mm}
Jason Poulos\thanks{\href{mailto:poulos@berkeley.edu}{\nolinkurl{poulos@berkeley.edu}}.}
\vspace{15mm}}

\date{\today}

%%%%%%%%%%%%%%%%%%%%%%%%%%%%%%%%%%%%%%%%%%%%%%%%%%
% Set document class
\documentclass[12pt]{article}

% Define packages
\usepackage{graphicx,amsfonts,psfrag,layout,subcaption,array,longtable,lscape,booktabs,dcolumn,natbib,amsmath,amssymb,amssymb,amsthm,setspace,epigraph,chronology,color, colortbl,caption,wasysym}
\usepackage[]{graphicx}\usepackage[]{color}
\usepackage[page]{appendix}
\usepackage{hyperref, url}
\usepackage[section]{placeins}
\usepackage[linewidth=1pt]{mdframed}
\usepackage[margin=1in]{geometry} %1 inch margins

% Footnotes stick at the bottom
\usepackage[bottom]{footmisc}

% New footnote characters
\usepackage{footmisc}
\DefineFNsymbols{mySymbols}{{\ensuremath\dagger}{\ensuremath\ddagger}\S\P
   *{**}{\ensuremath{\dagger\dagger}}{\ensuremath{\ddagger\ddagger}}}
\setfnsymbol{mySymbols}

% New tabular environment
\usepackage{tabularx}
\newcolumntype{Y}{>{\raggedleft\arraybackslash}X}% raggedleft column X

% Define appendix
\renewcommand*\appendixpagename{Appendix}
\renewcommand*\appendixtocname{Appendix}

% Position floats
\renewcommand{\textfraction}{0.05}
\renewcommand{\topfraction}{0.95}
\renewcommand{\bottomfraction}{0.95}
\renewcommand{\floatpagefraction}{0.35}
\setcounter{totalnumber}{5}

% Colors for highlighting tables
\definecolor{Gray}{gray}{0.9}

% Different font in captions
\newcommand{\captionfonts}{\scriptsize}

\makeatletter  % Allow the use of @ in command names
\long\def\@makecaption#1#2{%
  \vskip\abovecaptionskip
  \sbox\@tempboxa{{\captionfonts #1: #2}}%
  \ifdim \wd\@tempboxa >\hsize
    {\captionfonts #1: #2\par}
  \else
    \hbox to\hsize{\hfil\box\@tempboxa\hfil}%
  \fi
  \vskip\belowcaptionskip}
%\makeatother   % Cancel the effect of \makeatletter

% Set Spacing
%\doublespacing

%Theorem
\newtheorem{theorem}{Theorem}

% Number assumptions
\newtheorem*{assumption*}{\assumptionnumber}
\providecommand{\assumptionnumber}{}
\makeatletter
\newenvironment{assumption}[2]
 {%
  \renewcommand{\assumptionnumber}{Assumption #1}%
  \begin{assumption*}%
  \protected@edef\@currentlabel{#1}%
 }
 {%
  \end{assumption*}
 }
\makeatother

% Macros
\newcommand{\Adv}{{\mathbf{Adv}}}
\newcommand{\prp}{{\mathrm{prp}}}                  % How to define new commands
\newcommand{\calK}{{\cal K}}
\newcommand{\outputs}{{\Rightarrow}}
\newcommand{\getsr}{{\:\stackrel{{\scriptscriptstyle\hspace{0.2em}\$}}{\leftarrow}\:}}
\newcommand{\andthen}{{\::\;\;}}    %  \: \; for thinspace, medspace, thickspace
\newcommand{\Rand}[1]{{\mathrm{Rand}[{#1}]}}       % A command with one argument
\newcommand{\Perm}[1]{{\mathrm{Perm}[{#1}]}}
\newcommand{\Randd}[2]{{\mathrm{Rand}[{#1},{#2}]}} % and with two arguments
\newcommand{\E}{\mathrm{E}}
\newcommand{\ind}{\mathbb{I}} % Indicator function
\newcommand{\pr}{\mathbb{P}} % Generic probability
\newcommand{\ex}{\mathbb{E}} % Generic expectation
\newcommand{\Var}{\mathrm{Var}}
\newcommand{\Cov}{\mathrm{Cov}}
\newcommand{\cov}{\mathrm{Cov}}
\DeclareMathOperator*{\plim}{plim}
\newcommand\independent{\protect\mathpalette{\protect\independenT}{\perp}}
\def\independenT#1#2{\mathrel{\rlap{$#1#2$}\mkern2mu{#1#2}}}
\newcommand{\possessivecite}[1]{\citeauthor{#1}'s [\citeyear{#1}]}
\newcommand{\todo}[1]{{\color{red}{TO DO: \sc #1}}}

\begin{document}

\maketitle

\section{Motivation}

Methods of handling missing data for neural networks classification model

Given that we plan to use NNets for income prediction ($income \geq \$ 50K/yr$) on the Adult dataset, we must handle missing data. This is less problematic for ML models such as random forest, decision trees, etc.

%Include necessary ?background? information:
  % What is the application domain and/or field of research?
  % Why is the problem important?
  % What specific questions are you trying to answer?

Item nonresponse is a common problem in survey data in several domains. Several techniques for data imputation (replace missing values with plausible ones) and direct estimation(all missing data is analyzed using a maximum likelihood approach) have been developed \cite{de2003prevention}.

Proper statistical adjustement of missing data is very important, as naive solutions might introduce bias. We're interested in using the data to train neuronal networks. This must be taken into account becouse higher input values will result in a higher activation input.

In this project, we plan to evaluate different data imputation and direct estimation techniques within the context of using a neural network income classifier on the ADULT dataset. We plan to compare our results to previous techniques and models, such as bla bla bla, that addressed this same question.

\section{Data}

We plan to experiment with the Adult data set and then, given the results, move to the a larger sensus dataset.
\subsection{Adult data set}

\begin{tabular}{ l l | l l | l l}
  \hline
  Characteristics: & Multivariate &
  Observations: & 48842 &
  Area: & Social \\

  Features: & Categorical, Integer &
  Number of features: & 14 &
  Date Donated: & 1996-05-01 \\

  Associated Tasks: & Classification &
  Missing Values? & Yes &
  Number of Web Hits: & 559776 \\

  \hline
\end{tabular}

\subsection{Benchmarks}
% Table comparing accuracies for different models

\subsection{Exploratory data analysis}
% Table showing proportion of missing values for each feature

\section{Methods}

% Explain which methods you are planning to use and why.

\subsection{Techniques for handling missing data}
Assuming that the techniques below are easy to implemet, we would like to compare their efficiency in imputing the missing values.

\begin{enumerate}
\item Basic Statistics : Replace the missing data with the mean or median of the feature vector. This is the most naive approach and since the missing variables in the ADULT dataset are all categorical, using the mean is not appropriated.
\item One-hot : Create an binary variable to indicate whether or not a specific feature is missing. This technique was mentioned by Isabelle
\item Nearest Neighbor Imputation : Recursively compute the K-Nearest Neighbors of the observation with missing data and assign median of the K-neighbors to the missing data. This technique is used in airbnb's fraud detection algorithm and explained in their website.
\item Logistic Regression : train a logistic regression model with all features except the feature with the missing variable to predict the missing value.
\item Bagging : Use one bag tree model for each predictor based on all other
   predictors.
\item Factor analysis : Perform some sort of factorization on the design matrix, project the design matrix onto the first two eigen vectors and replace the missing values by the values that might be given by the projected design matrix.
\item Find other features with distribution similar to the feature containing missing data and use this information (e.g. correlation) to fill in in the missing data. However, if two features are highly correlated, it might be better to remove one of them.
\end{enumerate}

\subsection{Neural networks for classification}

\section{Anticipated results}

\pagebreak

%Bibliography
\bibliographystyle{plainnat}
\bibliography{refs}

%%Appendix
%\pagebreak
%\begin{appendices}
%
%\begin{figure}[htbp]
%\begin{center}
%\includegraphics[width = 1\textwidth]{rmse_ratec_rates}
%\caption{Simulated RMSE, binned by compliance rate and percent eligible for the RCT. Darker tiles correspond to worse estimates of PATT.}
%\label{fig:sim_tiles}
%\end{center}
%\end{figure}
%
%\end{appendices}

\itemize
\end{document}


